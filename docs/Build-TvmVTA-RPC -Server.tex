\documentclass{article}
\usepackage[utf8]{inputenc}
\usepackage{listings}

\title{Build TVM-VTA RPC Server}


\begin{document}

\maketitle


\section{Checkout tvm from github to DE10Pro board}
\begin{lstlisting}
Power On DE10Pro board, monitoring and wait until linux bootup via
sudo minicom -D /dev/ttyUSB0
within minicom get de10pro ip address by command:
ifconfig 
sshfs terasic@de10-pro-ip:/home/terasic mount-point/
cd mount-point/
mkdir workspace
cd workspace
git clone --recursive https://github.com/apache/tvm tvm

\end{lstlisting}
\section{Build tvm on de10pro board}
\subsection{Modify tvm/cmake/module/VTA.cmake}
\begin{lstlisting}
...
elseif(PYTHON)
  target_include_directories(vta SYSTEM PUBLIC 3rdparty)
  target_include_directories(vta SYSTEM PUBLIC
  "/usr/local/intelFPGA_lite/18.1/embedded/ds-5/sw/gcc/\
  arm-linux-gnueabihf/include")
  set (CMAKE_CXX_FLAGS "${CMAKE_CXX_FLAGS} -fpermissive")
elseif(${VTA_TARGET} STREQUAL "intelfocl")  # Intel OpenCL for FPGA rules
  target_include_directories(vta PUBLIC 3rdparty)
  set (CMAKE_CXX_FLAGS "${CMAKE_CXX_FLAGS} -std=c++17")

...
\end{lstlisting}

\subsection{run make on de10pro board}
In tvm root folder run below commands
\begin{lstlisting}
mkdir build
cd build
cmake ..
make -j4
\end{lstlisting}

\section{Setup and Start RPC}
After build tvm successfully, run below command to start RPC server on DE1-Pro board.
\subsection{Add De10pro Suppport in rpc\_server.py module }
Insert below code after line 76
\begin{lstlisting}
...
elif env.TARGET == "de10pro":
      # Load the de10pro program function.
      load_vta_dll()
...
\end{lstlisting}
\begin{lstlisting}
add  below line to ~/.bashrc
export PYTHONPATH=$PYTHONPATH:/usr/bin:Path totvm/tvm/python
source ~/.bashrc
pip3 install decorator
pip3 install cloudpickle
./apps/vta_rpc/start_rpc_server.sh
\end{lstlisting}


\end{document}